\documentclass[10pt,a4paper, landscape]{report}

\usepackage[utf8]{inputenc}
\usepackage{color}
\usepackage{colortbl}
\usepackage{xcolor}
\definecolor{gris}{rgb}{0.75,0.75,0.75}
\usepackage[top=1cm, bottom=2cm, left=0.5cm, right=0.5cm]{geometry}

\author{Kevin BASCOL, Bachir BOUACHERIA, Kevin LAOUSSING, Nicolas REYNAUD}
\title{Projet apprentissage automatique\\Rapport d'avancement $n^o 3$}
\date{\vfill 4 Mars 2015}


\parindent=0pt

\begin{document}

\maketitle

\section*{Avancement}
\begin{center}
	\bgroup
	\def\arraystretch{1.5}
	\begin{tabular}{|p{7cm}|p{1cm}|p{7cm}|p{2cm}|p{2cm}|p{7cm}|}
		\hline
		\rowcolor{gris}Tache & Date prévue & Accomplissement & Temps prévu & Temps pris & Commentaires\\
		\hline
		Représentation Graphique & 11/02 & Nombre de tests justes et faux en fonction de l'algo et du chiffre. & 3h & 3h & Cela nous permet de comparer les algorithmes et de savoir quels chiffres sont les mieux reconnus ou plus confondus. \\
		\hline
		Rapport d'avancement $n^o 3$ & 4/03 & En cours. & 1h & 20min & \\
		\hline
		Données apprentissages & 21/02 & Base terminée. & 10h & 5h & Chaque chiffre a environ 400 exemples.\\
		\hline
		\rowcolor{gris} & & & & & \\
		\hline
		Données de tests & 21/02 & En cours. & 10h & 1h & \\
		\hline
		Algorithme Réseau Neurone & 21/02 & En cours. & 8h & 1h & Recherche et compréhension d'API disponibles en ligne.\\
		\hline
	\end{tabular}
	\egroup
\end{center}

\section*{Remarques}
\begin{itemize}
\item Le planning en ligne a été mis à jour (voir http://ujm.eu5.org/aa/ )
\end{itemize}


\end{document}
