\documentclass[10pt,a4paper, landscape]{report}

\usepackage[utf8]{inputenc}
\usepackage{color}
\usepackage{colortbl}
\usepackage{xcolor}
\definecolor{gris}{rgb}{0.75,0.75,0.75}
\usepackage[top=1cm, bottom=2cm, left=0.5cm, right=0.5cm]{geometry}

\author{Kevin BASCOL, Bachir BOUACHERIA, Kevin LAOUSSING, Nicolas REYNAUD}
\title{Projet apprentissage automatique\\Rapport d'avancement $n^o 1$}
\date{\vfill 01 Février 2015}


\parindent=0pt

\begin{document}

\maketitle

\section*{Avancement}
\begin{center}
	\bgroup
	\def\arraystretch{1.5}
	\begin{tabular}{|p{7cm}|p{1cm}|p{7cm}|p{2cm}|p{2cm}|p{7cm}|}
		\hline
		\rowcolor{gris}Tache & Date prévue & Accomplissement & Temps prévu & Temps pris & Commentaires\\
		\hline
		Interface Graphique & 2/02 & Version minimale. & 5h & ~4h30 & Possibilité de rajouter des fonctions avancées (par exemple pour visualiser des données)\\
		\hline
		Fonction image vers matrice & 2/02 & Version avec réduction en haut et à gauche. & 1h & ~1h & \\
		\hline
		Code de Freeman & 2/02 & Version vue en cours. & 5h & ~2h & \\
		\hline
		Writer & 2/02 & Version avec un fichier principal et un fichier par matrice. & 1h & ~1h30 & Étudier la possibilité d'utiliser un seul fichier ou une base de donnée.\\
		\hline
		Reader & NP & Version renvoyant une liste complète chiffre/matrice/freeman. & NP & ~1h30 & \\
		\hline
		Rapport d'avancement $n^o 1$ & 2/02 & En cours. & 1h & 15min & Forme peut-être à revoir.\\
		\hline
		\rowcolor{gris} & & & & & \\
		Algorithme k-PPV & 6/02 & En cours. & 20h & ~2h & \\
		\hline
	\end{tabular}
	\egroup
\end{center}

\section*{Remarques}
Pour le moment nous sommes à peu près dans les temps. Mais, en prenant un peu plus de temps, de nombreuses améliorations sont envisageables.

\end{document}