\documentclass[10pt,a4paper, landscape]{report}

\usepackage[utf8]{inputenc}
\usepackage{color}
\usepackage{colortbl}
\usepackage{xcolor}
\definecolor{gris}{rgb}{0.75,0.75,0.75}
\usepackage[top=1cm, bottom=2cm, left=0.5cm, right=0.5cm]{geometry}

\author{Kevin BASCOL, Bachir BOUACHERIA, Kevin LAOUSSING, Nicolas REYNAUD}
\title{Projet apprentissage automatique\\Rapport d'avancement $n^o 2$}
\date{\vfill 14 Février 2015}


\parindent=0pt

\begin{document}

\maketitle

\section*{Avancement}
\begin{center}
	\bgroup
	\def\arraystretch{1.5}
	\begin{tabular}{|p{7cm}|p{1cm}|p{7cm}|p{2cm}|p{2cm}|p{7cm}|}
		\hline
		\rowcolor{gris}Tache & Date prévue & Accomplissement & Temps prévu & Temps pris & Commentaires\\
		\hline
		Interface Graphique & 2/02 & Version 2.1 & 5h & 6h & Ajout de graphiques sur les données et tests des codes de freeman.\\
		\hline
		Reader & 3/02 & Ajout de la possibilité de retourner une ligne particulière de la base. & 3h & 2h & Prévu depuis la v2 du planning. \\
		\hline
		Fonction code Freeman vers image & 2/02 & Version renvoyant une image de largeur double de l'image de base. & 1h & 1h & La plupart du temps a été utilisé pour essayer d'implémenter une version sans doubler la largeur de l'image. \\
		\hline
		kppv avec matrice & 7/02 & Utilisation des distances  Euclidienne, Manhattan, Chebyshev. Possibilité de paramétrer le nombre de voisin (3, 5 ou 7) & 4h & 4h & L'algorithme de kppv est beaucoup efficace avec la distance de Chebyshev. \\
		\hline
		kppv avec code freeman & 7/02 & Utilisation de la distance d'édition & 4h & 4h & L'algorithme reconnait tous les chiffres manuscrits \\
		\hline
		Distance Manatthan + Euclidienne + Chebyshev & 30/01 & Calcule des 3 distances vu en cours & 1h & 2h & La distance de Chebyshev, dans notre cas, est égale soit à 0 soit à 1 \\
		\hline
		Représentation Graphique & 11/02 & La base d'apprentissage est représentée sous forme d'histogramme & 3h & 3h & Cela nous permet de visualiser le nombre d'exemple appris par chiffre dans la base d'apprentissage. \\
		\hline
		Rapport d'avancement $n^o 2$ & 16/02 & En cours. & 1h & 10min & \\
		\hline
		\rowcolor{gris} & & & & & \\
		\hline
		Données apprentissages & 21/02 & En cours. & 10h & 1h & \\
		\hline
		Données de tests & 21/02 & En cours. & 10h & 1h & \\
		\hline
		Algorithme Réseau Neurone & 21/02 & Non commencé & 8h & & Algorithme seulement rapidement vu en cours . \\
		\hline
	\end{tabular}
	\egroup
\end{center}

\section*{Remarques}
\begin{itemize}
\item L'algorithme du réseau de neurones étant vu plus tard que prévu nous allons devoir décaler son implémentation.
\item Le planning en ligne a été mis à jour (voir http://ujm.eu5.org/aa/ )
\end{itemize}


\end{document}
